\documentclass{homework}

\usepackage[T1]{fontenc}
\usepackage{inconsolata}

\lstset{
  basicstyle=\ttfamily,
  mathescape,         
  literate={->}{$\rightarrow$}{2}
           {ε}{$\varepsilon$}{1}
}

\begin{document}

\title{Homework 4}
\author{Josh Bowden\vspace*{8pt}\\
CS 330 - Discrete Structures - Professor Sasaki}

\date{February 27, 2017}

\maketitle

%\noindent
%Prove the following statements. Show and explain \textbf{ALL} your work. Unless otherwise specified, give direct proofs.

%%
%% Problem 1
%%

\begin{problem}{1}[Let $P(n ) = 1^3 + 2^3 + \cdots + n^3 = \left(\frac{n(n+1)}{2}\right)^2$. Prove by induction that $\forall n \in \N \, (P(n))$.]

\begin{proof}[Theorem]
For every positive integer $n$,

\begin{flalign*}
& \sum_{j=1}^{k} j^3 = \left( \frac{n(n+1)}{2} \right)^2 && \qedhere
\end{flalign*}
\end{proof}

\begin{proof}
By induction on $n$,

\textbf{Base case:} $n = 1$

\begin{align*}
& \sum_{j=1}^{1} j^3 = \left( \frac{1(1+1)}{2} \right)^2 \then 1^3 = \left( \frac{2}{2} \right)^2 \then 1 = 1
\end{align*}

$\therefore \, \sum_{j=1}^{1} j^3 = \left( \frac{1(1+1)}{2} \right)^2$

\textbf{Inductive step:} Suppose that for positive integer $k$, $\sum_{j=1}^{k} j^3 = \left( \frac{n(n+1)}{2} \right)^2$, we will show that

\begin{align*}
\sum_{j=1}^{k+1} j^3 = \left( \frac{(k+1)(k+2)}{2} \right)^2
\end{align*}

Starting with the left side of the equation:

\begin{align*}
\sum_{j=1}^{k+1} j^3 &= \left( \sum_{j=1}^{k} j^3 \right) + (k + 1)^3 && \tt{by separating out the last term} \\
&= \left( \frac{k(k+1)}{2} \right)^2 + (k + 1)^3 && \tt{by the inductive hypothesis} \\
&= \frac{k^2 (k+1)^2}{4} + \frac{4 (k+1)^3}{4} \\
&= \frac{k^2 (k+1)^2 + 4 (k+1)^3}{4} \\
&= \frac{(k+1)^2 \left(k^2+4 (k+1)\right)}{4} \\
&= \frac{(k+1)^2 \left(k^2+4 k+4\right)}{4} \\
&= \frac{(k + 1)^2 (k + 2)^2}{4} \\
&= \left( \frac{(k+1)(k+2)}{2} \right)^2 && \tt{by algebra}
\end{align*}

$\therefore \, \sum_{j=1}^{k+1} j^3 = \left( \frac{(k+1)(k+2)}{2} \right)^2$
\end{proof}
\end{problem}


%%
%% Problem 2
%%

\begin{problem}{2}[Determine whether each of the proposed definitions is a valid recursive definition of a function $f$ from the set of non-negative integers to the set of integers. If $f$ is well defined, find a formula for $f(n)$ when $n$ is a non-negative integer and prove that your formula is valid.]

\begin{enumerate}[(a)]

%%
%% Problem 2(a)

\item $f(0) = 1, f(n) = -f(n - 1)$ for $n \geq 1$

This is a valid recursive definition of a function $f$.

A valid formula for $f(n)$ is

\begin{align*}
& f(n) = (-1)^n
\end{align*}

\begin{proof}
By induction on $n$,

\textbf{Base case:} $n = 0$

\begin{align*}
f(0) &= (-1)^0 \\
1 &= 1
\end{align*}

$\therefore \, f(0) = (-1)^0 = 1$

\textbf{Inductive step:} Suppose $f(n) = -f(n - 1) = \left(-1\right)^n$ for $n \geq 1$, we will show that:

\begin{align*}
f(n) = -f(n - 1) = {(-1)}^n
\end{align*}

Since,

\begin{align*}
f(n) &= -f(n - 1) \\
&= -(-1)^{(n-1)} && \tt{by the inductive hypothesis} \\
&= -\frac{{(-1)}^n}{(-1)^1} \\
&= (-1)^n
\end{align*}

$\therefore \, f(n) = \left(-1\right)^n $ for $n \geq 0$.
\end{proof}


%%
%% Problem 2(b)

\item $f(0) = 1, f(1) = 0, f(2) = 2, f(n) = 2f(n - 3)$ for $n \geq 3$

This is a valid recursive definition of a function $f$.

A valid formula for $f(n)$ is

\begin{align*}
& g(n) =
  \begin{cases}
   1 & n = 0 \\
   0 & (n \bmod 3) = 1 \\
   2^{\left\lfloor (n-1)/3\right\rfloor +1} & (n > 1) \text{ and } ((n \bmod 3) \neq 1)
  \end{cases}
\end{align*}

\begin{proof}
By induction on $n$,

\textbf{Base case:} $n = 0$\\
\begin{align*}
f(0) = g(0) \then 1 = 1 && g(0) \implies (n = 0) \equiv (0 = 0) \equiv \text{T}
\end{align*}
$\therefore \, f(0) = 1 = g(0)$

\textbf{Base case:} $n = 1$\\
\begin{align*}
f(1) &= g(1) \\
0 &= 0 && g(1) \implies ((n \bmod 3) = 1) \equiv ((1 \bmod 3) = 1) \equiv (1 = 1) \equiv \text{T}
\end{align*}
$\therefore \, f(1) = 0 = g(1)$

\textbf{Base case:} $n = 2$. Show that $f(2) = 2 = g(n)$.\\
\begin{align*}
f(2) &= g(2) \\
&= 2^{\left\lfloor (n-1)/3\right\rfloor +1} & \text{  since } g(2) \implies ((n > 1) \land ((n \bmod 3) \neq 1)) \\
& & \equiv ((2 > 1) \land ((2 \bmod 3) \neq 1)) \\
& & \equiv (\T \land (2 \neq 1)) \\
& & \equiv (\T \land \T) \\
& & \equiv \T \\
&= 2^{\left\lfloor (2-1)/3\right\rfloor +1} \\
&= 2^{\left\lfloor 1/3\right\rfloor +1} \\
&= 2^{0+1} \\
&= 2
\end{align*}
$\therefore \, f(1) = 0 = g(1)$


\textbf{Inductive step:} Suppose $f(n) = g(n)$ for $n > 1 \text{ and } (n \bmod 3) \neq 1$, we will show that:

\begin{align*}
f(n) &= g(n) && \text{for } n > 1 \text{ and } (n \bmod 3) \neq 1 \\
2f(n - 3) &= 2^{\left\lfloor (n-1)/3\right\rfloor +1} && \text{for } n > 1 \text{ and } (n \bmod 3) \neq 1
\end{align*}

Since,

\begin{align*}
2f(n - 3) &= 2^{\left\lfloor (n-1)/3\right\rfloor +1}
\end{align*}

$\therefore \, f(n) = g(n)$ for $n > 1 \text{ and } (n \bmod 3) \neq 1$.
\end{proof}


%%
%% Problem 2(c)

\item $f(0) = 0, f(1) = 1, f(n) = 2f(n + 1)$ for $n \geq 2$

This is not a valid recursive definition since $f(n)$ refers to $f(n + 1)$ which will then refer to $f(n + 2)$ and so on.


%%
%% Problem 2(d)

\item $f(0) = 0, f(1) = 1, f(n) = 2f(n - 1)$ for $n \geq 1$

This is a valid recursive definition of a function $f$.

A valid formula for $f(n)$ is 

\begin{align*}
& g(n) =
  \begin{cases}
    0 & n = 0 \\
    2^{n-1} & n \geq 1
  \end{cases}
\end{align*}

\begin{proof}
By induction on $n$,

\textbf{Base case:} $n = 0$

\begin{align*}
f(0) &= g(0) \\
0 &= 0 && g(0) \implies (n = 0) \equiv (0 = 0) \equiv \text{T} 
\end{align*}
$\therefore \, f(0) = g(0) = 0$

\textbf{Base case:} $n = 1$

\begin{align*}
f(1) &= g(1) \\
1 &= 2^{n-1} && g(1) \implies (n = 0) \equiv (0 = 0) \equiv \text{T} \\
1 &= 2^{1-1} \\
1 &= 2^{0} \\
1 &= 1
\end{align*}
$\therefore \, f(1) = g(1) = 1$

\textbf{Inductive step:} Suppose $f(n) = g(n)$ for $n \geq 1$, we will show that:

\begin{align*}
f(k + 1) &= g(k + 1) = 2^{k}
\end{align*}

Since,

\begin{align*}
f(k + 1) &= g(k + 1) && \tt{by the inductive hypothesis} \\
&= 2^{(k + 1) -1} && g(k + 1) \implies (k + 1 \geq 1) \equiv (k \geq 0) \equiv \text{T}  \\
&= 2^{k} \\
\end{align*}
$\therefore \, f(n) = g(n)$ for $n \geq 1$.
\end{proof}
\end{enumerate}
\end{problem}

%%

%%
%% Problem 3
%%

\begin{problem}{3}[Give a recursive definition of the sequence $\{a_n\}, n = 1, 2, 3, \dots$ if]

\begin{enumerate}[(a)]

%%
%% Problem 3(a)

\item $a_n = 4n - 2$\\
\begin{align*}
f(n) = 4n - 6
\end{align*}

%%
%% Problem 3(b)

\item $a_n = 1 + (-1)^n$ \\
\begin{align*}
f(n) = 1 + (-1)^n
\end{align*}

%%
%% Problem 3(c)

\item $a_n = n(n + 1)$ \\
\begin{align*}
f(n) = n^2 - n
\end{align*}

%%
%% Problem 3(d)

\item $a_n = n^2$ \\
\begin{align*}
f(n) = n^2 - 2n + 1
\end{align*}

\end{enumerate}
\end{problem}


%%
%% Problem 4
%%

\begin{problem}{4}[Give a recursive definition]

%%
%% Problem 4(a)

\begin{enumerate}[(a)]

\item the set of odd positive integers. \\
\begin{align*}
f(1) = 1, f(n) = f(n - 1) + 2
\end{align*}

\end{enumerate}
\end{problem}


%%
%% Problem 5
%%

\begin{problem}{5}[Give a recursive algorithm for finding the maximum of a finite set of integers, making use of the fact that the maximum of $n$ is the larger of the last integer in the list and the maximum is the larger of last integer in the list and the maximum of the first $n-1$ integer in the list.]

\begin{lstlisting}
def max(xs: List<Int>, cur: Int) where xs.length >= 1:
    def f(xs: List<Int>, cur: Int) where xs.length >= 1:
       if xs.length == 0:
           return cur
       else:
          let x = xs[xs.length - 1]
          let rest = xs[0..(xs.length - 2)]
          if x > cur:
             return max(rest, x)
          else
             return max(rest, cur)
     return f(xs, xs[0])
\end{lstlisting}
\end{problem}


%%
%% Problem 6
%%

\begin{problem}{6}[Devise a recursive algorithm to find $a^{2^n}$, where $a$ is a real number and $n$ is a positive integer.]

Let $f(a, n)$ where $a \in \R$ and $n \in \N$ represent the expression $a^{2^n}$. \\
Let $\text{exp}(a, n)$ where $a \in \R$ and $n \in \N$ represent the expression $a^n$.

\begin{align*}
& \text{exp}(a, 1) = a \\
& \text{exp}(a, n) = a * \text{exp}(a, n - 1) \\
& f(a, n) = \text{exp}(a, \text{exp}(2, n))
\end{align*}
\end{problem}


%%
%% Problem 7
%%

\begin{problem}{7}[Give a recursive algorithm for computing $n*a$ ("n times a") using addition, where $n$ is a positive integer and $a$ is a real number.]

Let $f(n, a)$ where $n \in \N$ and $a \in \R$ represent the expression $n * a$.

\begin{align*}
& f(1, a) = a \\
& f(n, a) = a + f(n - 1, a)
\end{align*}
\end{problem}


%%
%% Problem 8
%%

\begin{problem}{8}[Write the algorithm and the loop invariant (for the outer loop) for the iterative version of Bubble Sort and prove all three cases (Initialization, Maintenance, Termination).]

\begin{lstlisting}
def bubble_sort(xs: List):
    let n = xs.length
    let didSwap = true
    while didSwap:
        for x in {0..n-1}:
            if xs[i] > xs[i + 1]:
                let temp = xs[i]
                xs[i] = xs[i+1]
                xs[i+1] = temp
                didSwap = true
\end{lstlisting}
\end{problem}

\end{document}