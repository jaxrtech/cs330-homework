\documentclass{homework}
\usepackage{listings}

\lstset{
  basicstyle=\ttfamily,
  mathescape,         
  literate={->}{$\rightarrow$}{2}
           {ε}{$\varepsilon$}{1}
}

\begin{document}

\title{Homework 3}
\author{Josh Bowden\vspace*{8pt}\\
CS 330 - Discrete Structures\\
Professor Sasaki}

\date{February 20, 2017}

\maketitle

%\noindent
%Prove the following statements. Show and explain \textbf{ALL} your work. Unless otherwise specified, give direct proofs.


\begin{problem}{1}[Describe an algorithm to find the longest word in an English sentence (where a sentence is a sequence of symbols, either a letter or a blank, which can then be broken into alternating words and blanks).]

\begin{lstlisting}
start := -1
end := -1
max := 0
counter := 0
For i := 0 to n
   symbol := $\texttt{sentence}_\texttt{i}$
   If (symbol is a letter)
      If (counter == 0)
         start = i
      End-If 
      counter := counter + 1
   Else // symbol must be a blank
      If (counter > max)
         max := counter
         end := i
      End-If
      counter := 0
   End-If
End-For
\end{lstlisting}

The longest word in the sentence will be the symbols including within indices \texttt{start} and \texttt{end} exclusively.
\end{problem}

%%

\begin{problem}{2}[The ternary search algorithm locates an element in a list of increasing integers by successively splitting the list into three sub-lists of equal (or as close to equal as possible) size, and restricting the search to the appropriate piece. Specify the steps of this algorithm.]

\begin{enumerate}[(1)]

\item Let $x$ be the integer to search for.

\item Let $L$ be a list of $n$ elements.

\item Let $\Delta n = n \char`\\ 3$ (where `$\char`\\$' denotes integer division)

\item Let $\Lambda_1$, $\Lambda_2$, and $\Lambda_3$ denote the three sub-lists of near equal length, \\
where $\Lambda_1$ is the elements of $L_0$ to $L_{\Delta n}$, \\
and $\Lambda_2$ is the elements of $L_{\Delta n + 1}$ to $L_{2 \Delta n}$, \\
and $\Lambda_3$ is the elements of $L_{2 \Delta n + 1}$ to $L_{n}$. \\

\item Compare the first last item in $\Lambda_1$ and the first item in $\Lambda_3$.

\item Repeat the process by restarting at Step 1 where $L = \Lambda_i$ with the containing $\Lambda$ list.
\end{enumerate}

\end{problem}

%%

\begin{problem}{3}[Use the definition of Big-O to show that $(x^2 + 2x) / (2x + 1)$ is $O(x^2)$.]

\begin{proof}
Given,

\begin{flalign*}
f(x) &= (x^2 + 2x) / (2x + 1) \\
g(x) &= x^2
\end{flalign*}

By definition, for $f(x)$ to be $O(g(x))$, $f(n) \leq c \cdot g(n)$ for positive constants $c$ and $n_0$ such that $n \geq n_0$.

Then,

\begin{flalign*}
& \frac{f(n)}{g(n)} < 1 \\
& \implies \frac{(n^2 + 2n) / (2n + 1)}{n^2} < 1 \\
& \implies \frac{n^2 + 2n}{n^2(2n + 1)} < 1 \\
& \implies \frac{n^2}{n^2(2n + 1)} + \frac{2n}{n^2(2n + 1)} < 1 \\
& \implies \frac{1}{2n + 1} + \frac{2}{2n^2 + n} < 1 \tt{ for } n > 1
\end{flalign*}

$\therefore (x^2 + 2x) / (2x + 1)$ is $O(x^2)$.

\end{proof}
\end{problem}

%%

\begin{problem}{4}[Show that $x \log x$ is $O(x^2)$ but that $x^2$ is not $O(x \log x)$.]

\everypar{\setlength\hangindent{1em}}
\begin{proof}$O(x \log x) = O(x^2)$

Under a similar premise from the previous proof,

\begin{flalign*}
& \frac{n \log n}{n^2} < 1 \\
& \implies \frac{\log n}{n} < 1 \tt{ for } n > 1
\end{flalign*}

$\therefore x \log x$ is $O(x^2)$.
\end{proof}

\begin{proof}$O(x^2) \neq O(x \log x)$

\begin{flalign*}
& \frac{x^2}{x \log x} < 1 \\
& \implies \frac{1}{\log x} \nless 1 \tt{ for any positive } n
\end{flalign*}

$\therefore x^2$ is not $O(x \log x)$.
\end{proof}
\end{problem}

%%

\begin{problem}{5}[Show that each of these pairs of functions are of the same order.]

\begin{enumerate}[(a)]
\item $\log_{10}(x^2 + 1), \log_2 x$ \\

\begin{proof} $O(\log_{10}(x^2 + 1)) = O(\log_2 x) = O(n \log_10 n)$

Since, $\log_{10}(x^2 + 1) \nless 1 \tt{ for any positive } n$, we'll assume the order of $O(n \log n)$.

Then,

\begin{flalign*}
& \frac{ \log_{10}(n^2 + 1) }{n \log_{10} n} < 1 \tt{ for a positive } n
\end{flalign*}

$\therefore \log_{10}(x^2 + 1)$ is $O(n \log_{10} n)$.

Given that the Change of Base Formula is,

\begin{flalign*}
& \log_a x = \frac{\log_b x}{\log_b a}
\end{flalign*}

Then, $\log_2 x = \frac{\log_{10} x}{\log_{10} 2}$.

So,

\begin{flalign*}
& \frac{ \frac{\log_{10} n}{\log_{10} 2} }{n \log_{10} n} \\
& \implies \frac{1}{n (\log_{10} 2)} < 1 \tt{ for a positive } n.
\end{flalign*}

$\therefore \log_{10}(x^2 + 1)$ is $O(n \log n)$.

$\therefore O(\log_{10}(x^2 + 1)) = O(\log_2 x) = O(n \log_{10} n)$.
\end{proof}

\item $\log_{10} x, \log_2 x$
\end{enumerate}

\end{problem}

%%

\begin{problem}{6}[(a) Given a real number $x$ and a positive integer $k$, determine the number of multiplications used to find starting with $x$ and successively squaring (to find $x^2, x^4, x^8$, and so on). (b) Is this a more efficient way to find $x$\tt{\^{}}(2\tt{\^{}}k) (where \^{} means exponentiation) than by multiplying $x$ by itself the appropriate number of times?]

\begin{enumerate}[(a)]
\item $log_2(k)$ operations of successive squaring
\item Yes, since multiplying $x$ by itself takes $k$ operations
\end{enumerate}
\end{problem}

%%

\begin{problem}{7}[Determine the least number of comparisons (the best-case performance) needed]

\begin{enumerate}[(a)]
\item To find the maximum of a sequence of $n$ integers. \\
$O(n)$ assuming an unsorted list since all integers within the list must be tested to ensure the maximum integer was obtained.

\item To locate an element in a list of n terms using a binary search. \\
$O(1)$ if the element is located on the first iteration.
\end{enumerate}
\end{problem}

%%


\begin{problem}{8}[Describe how the number of comparisons used in the worst case changes when the size of the list to be sorted
doubles from $n$ to $2n$, where $n$ is a positive integer when these sorting algorithms are used.]

\begin{enumerate}[(a)]
\item The selection sort begins by finding the least element in the list. This element is swapped to the front. Then the least
element among the remaining elements is found and swapped into the second position. This procedure is repeated
until the entire list has been sorted.\\
$n^2$ to $4n^2$

\item The binary insertion sort is a variation of the insertion sort that uses a binary search technique rather than a linear
search technique to identify the location to insert the $i$'th element in the correct place among the previously sorted
elements. \\
$n^2$ to $4n^2$
\end{enumerate}

\end{problem}


\end{document}