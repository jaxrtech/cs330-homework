\documentclass{homework}

\begin{document}

\title{Homework 2}
\author{Josh Bowden\vspace*{8pt}\\
CS 330 - Discrete Structures\\
Professor Sasaki}

\date{February 4, 2017}

\maketitle

\noindent
Prove the following statements. Show and explain \textbf{ALL} your work. Unless otherwise specified, give direct proofs.


\begin{problem}{1}[The sum of two even integers is always even.]

\begin{proof}If $a$ and $b$ are even integers, then $a + b$ is an even integer.

Assume $a$ and $b$ are even integers.

Then by definition,\\
\begin{flalign*}
a &= 2m \text{ where } m \in \Z,\\
b &= 2n \text{ where } n \in \Z.
\end{flalign*}

Then,

\begin{flalign*}
a + b &= (2m) + (2n)\\
 &= 2(m + n),
\end{flalign*}

and $2(m + n)$ is an even integer by definition.

Therefore, $a + b$ is even by definition.
\end{proof}
\end{problem}

%%

\begin{problem}{2}[State the contrapositive of (for all integers $n$, if $n^2$ is odd, then $n$ is odd) and prove it.]

\begin{flalign*}
\text{Given} &: \forall n \in \Z \, (n^2 \text{ odd} \cond n \text{ odd})\\
\text{Contrapositive} &: \forall n \in \Z \, (n \text{ even} \cond  n^2 \text{ even} )
\end{flalign*}

\begin{proof} $\forall n \in \Z \, (n \text{ even} \cond  n^2 \text{ even})$

Let $n$ be an integer.

Assume $n$ is even.

Then by definition, $n = 2k$ where $k \in \Z$.

So,

\begin{flalign*}
n^2 &= (2k)^2 
&= (4)(k \cdot k)
&= 2k \cdot 2k.
\end{flalign*}

And two even integers multiplied are always even.

Therefore, $n^2$ is even.
\end{proof}
\end{problem}

%%

\break

\begin{problem}{3}[Prove that if the sum of the digits of a 3-digit number $n$ is divisible by 9, then $n$ is divisible by 9.]

\begin{proof}If the sum of the digits of a 3-digit number $n$ is divisible by 9, then $n$ is divisible by 9.

Let $n$ be an integer.

Assume that $n$ is a 3-digit number such that $n \geq 100 \land n \leq 999$.

$n$ can be represented by its digits as,

\begin{flalign*}
n &= n_3(100) + n_2(10) + n_1(1).
\end{flalign*}

Then, by rearranging the previous expression,

\begin{flalign*}
n &= n_3(1 + 99) + n_2(1 + 9) + n_1 && 100 = 1 + 99\\
&= n_3 + n_3(99) + n_2 + n_2(9) + n_1 && \text{Distributive property of multiplication}\\
&= n_3(99) + n_2(9) + n_3 + n_2 + n_1 && \text{Associative property of addition}.
\end{flalign*}

Since divisibility is defined as $a/b={c}$ where $a, b, c \in \Z$, and

\begin{flalign*}
\frac{n_3(99)}{9} &= n_3(11) \text{ and } n_3(11) \in \Z,\\
\frac{n_2(9)}{9} &= n_2(1) \text{ and } n_2(1) \in \Z,
\end{flalign*}

so the terms $n_3(99)$ and $n_2(9)$ must be divisible by 9.

Then, $n$ is divisible by 9 if the remaining terms ($n_3, n_2, n_1$) sum such that,\\
\begin{flalign*}
\frac{n_3 + n_2 + n_1}{9} = k \; (\text{where } k \in \Z)
\end{flalign*}

Then, if all terms are divisible by 9, the sum of those terms must be divisble by 9.

Therefore, if $(n_3 + n_2 + n_1) / 9 = k \; (\text{where } k \in \Z)$, then $n$ is divisible by 9.
\end{proof}
\end{problem}

%%

\break

\begin{problem}{4}[Prove by contradiction that the product of two odd numbers is odd.]

\everypar{\setlength\hangindent{1em}}
\begin{proof}$(\forall m \in \Z) (\forall n \in Z) \, ((m \text{ odd} \land n \text{ odd}) \cond ((m \cdot n) \text{ odd}))$ 

Let $m$ and $n$ be integers.

Assume $m$ and $n$ are odd.

To prove by contradition, assume $m \cdot n$ is even.

Then by definition, $mn = 2k$ for some integer $k$.

So,

\begin{flalign*}
2k + 1 &= mn\\
2k &= mn - 1\\
k &= \frac{mn - 1}{2}\\
k &= mn - \frac{1}{2}
\end{flalign*}

Then, $k$ is a rational number and not an integer ($k \in \mathbb{Q} \land k \not\in \Z$), \\
since $mn$ is an integer, given that the product of two integers is an integer,\\
and subtracting $1/2$ from an integer will result in a rational number.

This is a contradition since $k$ was assumed to be an integer.

Therefore, $m \cdot n$ cannot be even.
\end{proof}

\end{problem}

%%

\begin{problem}{5}[Prove that the product of two rational numbers is rational.]

\begin{proof}$(\forall p \in \Q) (\forall q \in \Q) \, ((p \cdot q) \in \Q)$

Let $p$ and $q$ be rational numbers.

Then by definition,

\begin{flalign*}
p &= \frac{a}{b}\\
q &= \frac{c}{d},
\end{flalign*}

where $\{a, b, c, d\} \in \Z$.

Then,

\begin{flalign*}
p \cdot q &= \frac{a}{b} \cdot \frac{c}{d} \\
&= \frac{ab}{cd}.
\end{flalign*}

Then $\{ab, cd\} \in \Z$ since the product of two integers is an integer.

So then, $\frac{ab}{cd}$ is a rational number.

Therefore, the product of two rational numbers is rational.
\end{proof}

\end{problem}

\end{document}