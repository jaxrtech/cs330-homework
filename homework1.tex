\documentclass[11pt]{article}
 
\usepackage[margin=0.75in, nomarginpar]{geometry} 
\usepackage{layout}
\usepackage[fleqn]{amsmath}
\usepackage{amsthm,amssymb}
\usepackage{xparse}
\usepackage[shortlabels]{enumitem}
\usepackage{titling}
\usepackage{etoolbox}

%%

\NewDocumentEnvironment{problem}{s O{Problem} m o}
  {\begin{trivlist}
     \item[\hskip \labelsep {\bfseries #2}\hskip \labelsep {\bfseries #3.}]
     \IfValueTF{#4}
       {#4 \IfBooleanT{#1}{\hspace*{\fill}\\}}
       {\hspace*{\fill}\\}
  }
  {\end{trivlist}}

\newenvironment{quoted}
  {\quote\makebox[0pt][r]{``}\ignorespaces}
  {\unskip''\endquote}

\let\tt\text
\let\bicond\leftrightarrow
\let\cond\rightarrow

\setlength{\topskip}{0cm}
\setlength{\droptitle}{-1.25cm}

% restart equation numbering each block
% http://tex.stackexchange.com/a/261982/44533
\preto\equation{\setcounter{equation}{0}}
\makeatletter
\pretocmd\start@gather{\setcounter{equation}{0}}{}{}
\pretocmd\start@align{\setcounter{equation}{0}}{}{}
\pretocmd\start@multline{\setcounter{equation}{0}}{}{}
\makeatother

%%

\begin{document}

\title{Homework 1}
\author{Josh Bowden\vspace*{8pt}\\
CS 330 - Discrete Structures\\
Prof Sasaki}

\date{January 27, 2016}

\maketitle

\begin{problem}*{1}[Express each of these propositions as an English sentence.]
$p$: I bought a lottery ticket this week \\
$q$: I won the million-dollar jackpot

\begin{enumerate}[(a)]
\item $\neg p \\ \equiv$ I did not buy a lottery ticket this week.

\item $p \land q \\ \equiv$ I bought a lottery ticket this week, and I won the million-dollar jackpot.

\item $\neg p \land \neg q \\ \equiv$ I did not buy a lottery ticket this week, and I did not win the million-dollar jackpot.

\item $p \lor q \\ \equiv$ Either, I bought a lottery ticket this week, or I won the million-dollar jackpot.

\item $p \bicond q \\ \equiv$ I bought a lottery ticket this week if and only if I won the million-dollar jackpot.

\item $\neg p \lor (p \land q) \\ \equiv$ Either, I did not buy a lottery ticket this week, or not only did I buy a lottery ticket this week but also won the million-dollar jackpot.

\item $p \cond q \\ \equiv$ If I bought a lottery ticket this week, then I won the million-dollar jackpot.

\item $\neg p \cond \neg q \\ \equiv$ If I did not buy a lottery ticket this week, then I did not win the million-dollar jackpot.

\end{enumerate}
\end{problem}

\vspace*{0pt}

\begin{problem}{2}[Write each of these statements in the form ``if $p$, then $q$'' in English.]
\begin{enumerate}[(a)]
\item It is necessary to wash the boss's car to get promoted.\\
  $\equiv$ If you got promoted, then you washed the boss's car.
  
\item Winds from the south imply a spring thaw.\\
  $\equiv$ If there are winds from the south, then there is a spring thaw.
  
\item A sufficient condition for the warranty to be good is that you bought the computer less than a year ago.\\
  $\equiv$ If the warranty is good, then you bought the computer less than a year ago.
  
\item Willy gets caught whenever he cheats.\\
  $\equiv$ If Willy cheats, then he gets caught.
  
\end{enumerate}
\end{problem}

\vspace*{0pt}

\begin{problem}{3}[Are these system specifications consistent?]

\begin{quoted}
  Whenever the system software is being upgraded, users cannot access the file system.\\
  If users can access the file system, then they can save new files.\\
  If users cannot save new files, then the system software is not being upgraded.
\end{quoted}

\vspace*{-5mm}

\begin{flalign*}
& \tt{upgrade} \cond \neg \tt{access} \\
& \tt{access} \cond \tt{save}  \tag*{Hypothesis} \\
& \neg \tt{save} \cond \neg \tt{upgrade} \\
%
& \equiv
  (\tt{upgrade} \cond \neg \tt{access}) 
  \land (\tt{access} \cond \tt{save})
  \land (\neg \tt{save} \cond \neg \tt{upgrade})  \tag*{Hypothesis (rewritten)} \\
%
& \equiv
  (\neg \tt{upgrade} \lor \neg \tt{access})
  \land (\neg \tt{access} \lor \tt{save})
  \land (\neg \neg \tt{save} \lor \neg \tt{upgrade}) \tag*{Conditional identity} \\
%
& \equiv
  (\neg \tt{upgrade} \lor \neg \tt{access})
  \land (\neg \tt{access} \lor \tt{save})
  \land (\tt{save} \lor \neg \tt{upgrade}) \tag*{Double negation law} \\
\end{flalign*}

No, the system specification is \textit{not} consistent.

\end{problem}

\vspace*{0pt}

\begin{problem}{4}[Without using truth tables, show that $p \bicond q$ and $(p \bicond q) \land (q \bicond p)$ are logically equivalent.]

$p \bicond q \equiv (p \bicond q) \land (q \bicond p)$ by the Conditional identity

\end{problem}

\vspace*{0pt}

\begin{problem}{5}[Determine if the following argument is correct or not and show in detail why.]
\begin{quoted}
If I play baseball, then I am sore. I use the swimming pool if I am sore. I did not use the swimming pool.
Therefore, I did play baseball.
\end{quoted}

This argument is not correct, because even though they did not use the swimming pool, there is no implication that they were sore to imply that they played baseball.

\end{problem}

\vspace*{0pt}

\begin{problem}{6}
Let $C(x)$ be the statement ``x has a cat'', let $D(x)$ be the statement ``x has a dog'', and $F(x)$ be the statement ``x has a ferret''. \\
Express each of these statements in terms of $C(x)$, $D(x)$, $F(x)$, quantifiers, and logical connectives. \\
Let the domain consist of all students in your class.

\begin{enumerate}[(a)]
\item A student in your class has a cat, a dog, and a ferret. \\ 
  $\equiv \exists x \, (C(x) \land D(x) \land F(x))$ 

\item All students in your class have a cat, a dog, and a ferret. \\ 
  $\equiv \forall x \, (C(x) \land D(x) \land F(x))$ 

\item Some student in your class has a cat and a ferret, but not a dog. \\ 
  $\equiv \exists x \, (C(x) \land F(x) \land \neg D(x) )$ 
  
\item No student in your class has a cat, a dog, or a ferret. \\ 
  $\equiv \neg \exists x \, (C(x) \lor D(x) \lor F(x))$ 
  
\item For each of the three animals, cats, dogs, and ferrets, there is a student in your class who has this animal as a pet. \\ 
  $\equiv (\exists x \, C(x)) \land (\exists x \, D(x)) \land (\exists x \, F(x))$
  
\end{enumerate}
\end{problem}

\vspace*{0pt}

\begin{problem}{7}[Determine the truth value of each of these statements if the domain of each variable consists of all real numbers.]

\begin{enumerate}[(a)]
\item $\exists x \, (x^2 = 2) \\ \equiv \tt{T} \ (\textrm{since } x = \sqrt{2})$
\item $\exists x \, (x^2 = -1) \\ \equiv \tt{F} \ (\textrm{since } x \not\in \mathbb{I})$
\item $\forall x \, (x^2 + 2 \geq 1) \\ \equiv \tt{T} \ (\textrm{since } x^2 \geq -1)$
\item $\forall x \, (x^2 \neq x) \\ \equiv \tt{F} \ (\textrm{since } x = 1)$
\end{enumerate}
\end{problem}

\vspace*{0pt}

\begin{problem}{8}
Let $Q(x, y)$ be the statement ``student x has been a contestant on quiz show y.'' 
Express each of these sentences in terms of $Q(x, y)$, quantifiers, and logical connectives where the domain for x consists of all students at your school and the domain y consists of all quiz shows on television.

\begin{enumerate}[(a)]
\item There is a student at your school who has been a contestant on a television quiz show. \\
  $\equiv \exists x \, \exists y \ Q(x, y)$

\item No student at your school has ever been a contestant on a television quiz show. \\
  $\equiv \neg (\exists x \, \exists y \ Q(x, y))$

\item There is a student at your school who has been a contestant on Jeopardy and on Wheel of Fortune. \\
  $\equiv \exists x \ (Q(x, \textrm{Jeopardy}) \land Q(x, \textrm{Wheel of Fortune}))$

\item Every television quiz show has had a student from your school as a contestant. \\
  $\equiv \forall y \, \exists x \ Q(x, y)$

\item At least two students from your school have been a contestant on Jeopardy. \\
  $\equiv \exists x_1 \, \exists x_2 \ (\neg (x_1 \bicond x_2) \land Q(x_1, \textrm{Jeoparty}) \land Q(x_2, \textrm{Jeoparty}))$

\end{enumerate}

\end{problem}

\vspace*{0pt}

\begin{problem}{9}
For each of these sets of premises, what relevant conclusion or conclusions can be drawn? \\
Explain the rules of inference used to obtain each conclusion from the premises.

\begin{enumerate}[(a)]
\item
  ``If I play hockey, then I am sore the next day.'' \\
  ``I use the whirlpool if I am sore.'' \\
  ``I did not use the whirlpool.''

\vspace*{-7mm}

\begin{flalign}
& \tt{play hockey} \cond \tt{sore} \\
& \tt{sore} \cond \tt{use whirlpool} \\
& \neg (\tt{use whirlpool}) \\[-5pt]
& \cline{1-1}
& \therefore \tt{play hockey} \cond \tt{use whirlpool}  \hspace{0.5cm}\text{(Hypothetical syllogism, 1, 2)} \\
& \therefore \neg (\tt{play hockey})  \hspace{0.5cm}\text{(Modus tollens, 4, 3)}
\end{flalign}


\item
  ``If I work, it is either sunny or partly sunny.'' \\
  ``I worked last Monday or I worked last Friday.'' \\
  ``It was not sunny on Tuesday.'' \\
  ``It was not partly sunny on Friday.''
  
\vspace*{-7mm}

\begin{flalign}
& \forall day \, (%
    \tt{Work}(day) \cond (%
      \tt{Weather}(day, \text{sunny})
      \lor \tt{Weather}(day, \text{partly sunny})
    )%
  )%
  \\
& \tt{Work}(\text{Monday}) \lor \tt{Work}(\text{Friday}) \\
& \neg \tt{Weather}(\text{Tuesday}, \text{sunny}) \\
& \neg \tt{Weather}(\text{Friday}, \text{partly sunny}) \\[-5pt]
& \cline{1-1}
& \therefore
    \tt{Weather}(\text{Friday}, \text{sunny}) \hspace{0.5cm}\text{(Disjunctive syllogism, 1, 4)}
\end{flalign}


\item
  ``All insects have six legs.'' \\
  ``Dragonflies are insects.'' \\ 
  ``Spiders do not have six legs. \\
  ``Spiders eat dragonflies.''
  
\vspace*{-7mm}
  
\begin{flalign}
& \forall x \, (\tt{Insect}(x) \cond (\tt{Legs}(x) = 6)) \\
& \tt{Insect}(\tt{dragonfly}) \\
& \tt{Legs}(\tt{spider}) \neq 6 \\
& \tt{Eat}(\tt{spider}, \tt{dragonfly}) \\[-5pt]
& \cline{1-1}
& \therefore
    \tt{Legs}(\tt{dragonfly}) = 6  \hspace{0.5cm}\text{(Modus ponens, 1, 2)} \\
& \therefore
    \neg \tt{Insect}(\tt{spider})  \hspace{0.5cm} \text{(Modus tollens, 1, 3)}
\end{flalign}

\item
  ``Every student has an Internet account.'' \\
  ``Homer does not have an Internet account.'' \\
  ``Maggie has an Internet account.''

\vspace*{-7mm}

\begin{flalign}
& \forall x \, (\tt{Student}(x) \cond \tt{Internet}(x)) \\
& \neg \tt{Internet}(\tt{Homer}) \\
& \tt{Internet}(\tt{Maggie}) \\[-5pt]
& \cline{1-1}
& \therefore
    \neg \tt{Student}(\tt{Homer})  \hspace{0.5cm}\text{(Modus tollens, 1, 2)} \\
& \therefore
    \tt{Student}(\tt{Maggie})  \hspace{0.5cm}\text{(Modus tollens, 1, 3)}
\end{flalign}

\end{enumerate}
\end{problem}

\end{document}